% 12 Oct 90 LaTeX figure boxing macros  by Jack Snoeyink
%                                       w/ ideas from John Hershberger
%                                       and code from Tom Rokicki
%%%%%%%%%%%%%%%%%
% 
% This file contains two classes of macros for dealing with figures in
% Latex documents:  
%  \figbox (and its relatives) take a box containing a figure or table
%    or whatever, caption it, and either place it on the page and wrap the
%    text around it or let it float (as usual).
%  \boxeps and \boxps take encasulated or plain postscript from a file
%    and put it in a box for use by \figbox
%
%%%%%%%%%%%%%%%%
%
% The most important macro for dealing with boxes is:
%
% \figbox: caption a box as a figure or table and place it in the margin
%       (a marginal box) or float it (a floating box).
%       Wide marginal boxes are promoted to one column floats; wide one
%       column floats are promoted to two column floats.
%
% Usage: \figbox [*htblr] {box} {style} {label} {caption}
%
% The optional argument default is [htb].  Other choices
%   * two column (must come first),
%   h here, t top, b bottom (as in latex),
%   l left, r right (marginal boxes that do not float)
%
% NB: Left and right marginal boxes will not overlap in height.
%
% Two column figures default to htb unless some other 
%       options follow the *.
%   They can also be specified by style {figure*} or {table*}.
%
% box is a box containing a figure or table.
% style is the caption style {figure}, {figure*}, {table}, or {table*}.  
% label is the label assigned.
%       if empty, the figure will not be labeled.
% caption is of the form {}, {Caption} or {[short caption] Long caption},
%       where the short caption goes into the list of figures/tables.
%       if empty, the figure will not be captioned or labeled.
% 
% The lower level macro \captionbox, described much later, can be used
% to combine two or more captioned figures into a single marginal or
% floating box.
%
%%%%%%%%%%%%%%%%
%
% options defined by macros are: \captionstyle \boxcaptionstyle
% \maxfigfraction \figboxhang \figboxmargin 
%
% \captionstyle is executed for all floating box captions
% \boxcaptionstyle is executed for all marginal box captions.
%       Default \raggedright
% \maxfigfraction is the maximum fraction of a column that can be taken up
%       by a marginal box before it gets converted to a float.
%       Default .6
% dimen \figboxhang is the distance a marginal box hangs into the margin.
%       Default 0pt
% dimen \figboxmargin is the width of space between the text and a
%       marginal box. 
%       Default .15in
%
%%%%%%%%%%%%%%%%

%%%%%%%%%%%%%%%%
%
% The most important macros for turning postscript into boxes are:
% 
% \boxeps: Form a box containing encapsulated postscript from a file at a
%       given scale.
% \boxps: Form a box containing postscript from a file at a given scale, 
%       width, and height.
% 
%
% Usage: \boxeps[local commands] {file} {scale}
%        \boxps[local commands] {file} {scale} {width} {height}
%
% The optional local commands are executed after scaling. As described
%       below, they are useful for setting origins, non-uniform scaling, etc?
% file is the filename in the \figuredirectory and without the .ps
%       extension.
% scale is a real number: 1 to use the drawn size.  0.9 to use 90% of
%       drawn size.  Can be changed by local commands. (see below)
% width and height are dimensions of the box produced by \boxps.
%
%%%%%%%%%%%%%%%%
%
% options: \DVIscaling \figuredirectory \missingfigure \boxer
% local commands: \horizscale \vertscale
%
% \DVIscaling is the number the dvi to ps converter considers a unit
%       scaling transformation.  Default: 1 for dvi2ps  --- 100 for
%       some of Rokicki's versions of dvips.
%
% \globalscaling is the scaling applied to every picture --- useful
%       for making double size figures for journals.  Default: 1
%
% \figuredirectory is the pathname where figure postscript files are kept.
%       Default: ./figure
%
% \missingfigure{file} is executed when the file is not found.
%       Default: \def\missingfigure#1{\hbox{Missing figure #1.ps}}
%
% \boxer is the command used by \boxps to make a box of the desired size
%       containing the figure.  Possible commands depend on the origin of
%       the figure, and include \llboxer \lcboxer \ulboxer and \ccboxer.
%       (ll = lower left, ul = upper left, cc = dead center, etc.)
%       \boxer can be defined in the local commands or globally.
%
% Local commands \horizscale, and \vertscale can scale the picture
%       horizontally and vertically by different factors.  \scale
%       scales the picture uniformly.  Scales accumulate.
%
%%%%%%%%%%%%%%%%



%%%%%%%%%%%%%%%%
%
% Default options
%

\def\captionstyle{}
\def\boxcaptionstyle{\raggedright}

\def\maxfigfraction{.6}%        --- max frac of column for marg figure

\newdimen\figboxmargin%         --- margin between text and picture
\figboxmargin = .15in

\newdimen\figboxhang%           --- amount to hang picture into margin
\figboxhang = 0pt

\def\DVIscaling{1}%             --- for dvi2ps
\def\globalscaling{1}%          --- for scaling everything
\def\figuredirectory{./figures}%--- where to find the .ps files 
\let\boxer=\llboxer%            --- default origin in lower left for non-
%                                     encapsulated postscript figures

\def\missingfigure#1{\hbox{Missing figure #1.ps}}


%%%%%%%%%%%%%%%%
%
% Use latex internals
%
\catcode `\@=11
\special{! /@scaleunit 1 def }

\newbox\figurebox

\def\figbox{%   [place] box style label caption
\@ifnextchar[{\figboxaux}{\figboxaux[htb]}}

%
% Parse position parameters and try appropriate placement
%
\long\def\figboxaux[#1#2]#3#4#5#6{%     --- want left margin figure?
\writepict{{#3}{#4}{#5}{#6}}%           --- save figure if journalpicts
\setbox\figurebox\hbox{#3}%
\if l#1\tryleftbox{#4}{#5}{#6}%
\else%                                  --- want right margin figure?
\if r#1\tryrightbox{#4}{#5}{#6}%
\else%                                  --- want one or two column figure?
\if *#1\checktwocoloptions#2]{\box\figurebox}{#4*}{#5}{#6}%
\else\tryonecol[#1#2]{#4}{#5}{#6}%
\fi
\fi
\fi\ignorespaces}

%
% Try a placement; promote if too wide
%

\long\def\tryleftbox#1#2#3{%            --- does left margin figure fit?
\ifdim\wd\figurebox>\maxfigfraction\columnwidth \tryonecol[htb]{#1}{#2}{#3}%
\else\leftbox{\captionbox{\box\figurebox}{#1}{#2}{#3}}\fi}

\long\def\tryrightbox#1#2#3{%   --- does right margin figure fit?
\ifdim\wd\figurebox>\maxfigfraction\columnwidth \tryonecol[htb]{#1}{#2}{#3}%
\else\rightbox{\captionbox{\box\figurebox}{#1}{#2}{#3}}\fi}

\def\checktwocoloptions{%       --- make sure twocol has htb or p
\@ifnextchar]{\floatbox[htb}{\floatbox[}}

\long\def\tryonecol[#1]#2#3#4{% --- does one column figure fit?
\ifdim\wd\figurebox>\columnwidth \floatbox[#1]{\box\figurebox}{#2*}{#3}{#4}%
\else\floatbox[#1]{\box\figurebox}{#2}{#3}{#4}\fi}

%
% Do the work for a floating box.
%
% \floatbox[place]{box} {style} {label} {caption}
%

\long\def\floatbox[#1]#2#3#4#5{%
\begin{#3}[#1]
\hbox to \hsize{\hfil#2\hfil}
\captionandlabel{#3}{#4}{#5}
\end{#3}
}


%
% Make a caption the width of the box with given style. 
%
%\captionbox {box} {style} {label} {[optshortcap] caption}
%
% options: \boxcaptionstyle 
%
%%%%%%%%%%%%%%%%
%
% This can be used with box commands (\hbox, \vbox, \mbox, and the like),
% along with \figbox with an empty caption, to place two or more captioned
% boxes in a single marginal or floating box.   
%
% e.g. \figbox{\hbox{
%                    \captionbox{ \boxeps{fig1}1} {figure} {f1} {a figure}
%                    \hskip 1in
%                    \captionbox{ \boxeps{tab1}1} {table} {t1} {a table}
%                   }} {figure} {ignored-label} {}
%

\long\def\captionbox#1#2#3#4{% box style label {[optshortcap] caption}
\setbox\figurebox\hbox{#1}%
\parbox[t]{\wd\figurebox}{%
\bigskip\box\figurebox
\let\captionstyle=\boxcaptionstyle
\captionandlabel{#2}{#3}{#4}
\bigskip
}}

%                               --- caption and label if label is nonempty
\def\captionandlabel#1#2#3{%    --- {style} {label} {caption}
\def\testit{#3}%
\ifx\testit\empty\else%         --- if non empty caption
\writecapt{{#1}{#2}{#3}}%       --- save for journal figs
\captypeunstarred#1*.%          --- remove optional stars
\getcaption#3\endc@ption%       --- parse the caption
\def\testit{#2}%                --- check for empty label
\ifx\testit\empty\else\label{#2}\fi
\fi}

\def\captypeunstarred#1*#2.{%   --- remove two-column star from style
\def\@captype{#1}}

%
% Parse the caption of the form {caption} or {[short] long caption} 
%

\def\getcaption{\@ifnextchar[{\getcaptwo}{\getcapone}}
\long\def\getcapone#1\endc@ption{\caption[#1]%
{\def\baselinestretch{1}\Large\normalsize\captionstyle\ignorespaces #1}}
\long\def\getcaptwo[#1]#2\endc@ption{\caption[#1]%
{\def\baselinestretch{1}\Large\normalsize\captionstyle\ignorespaces #2}}

%
% Do the right thing with an already captioned marginal box
%
% \leftbox{box}
% \rightbox{box}
%
% options: \figboxmargin \figboxhang \figuredirectory

\newdimen\figboxht
\newcount\figboxn


\newcount\figboxlines%          --- picture height in number of lines
\newdimen\figboxwid%            --- picture width

\newif\ifisleftbox

\long\def\leftbox#1{%
\setbox\figurebox\hbox{#1}\global\isleftboxtrue
\startmarginbox
\vadjust{\smash{\rlap{\hskip\hsize\hskip\figboxhang
\llap{\raise.7\baselineskip\box\figurebox\hskip\rightskip}}}}%
\endmarginbox%
}

\long\def\rightbox#1{%
\setbox\figurebox\hbox{#1}\global\isleftboxfalse
\startmarginbox
\smash{\llap{\raise.7\baselineskip\box\figurebox\hskip\figboxmargin}}%
\endmarginbox%
}

\def\startmarginbox{%
\ifvmode\passpict\let\endmarginbox=\indent
\else\message{WARNING: marginbox in not in vmode}\hfilneg\ \passpict
\let\endmarginbox=\relax\fi
%
\figboxht=\dp\figurebox%         actual height
\advance\figboxht by 1.3\baselineskip% round up
\vskip.95\figboxht\penalty-300\vskip-.95\figboxht% filbreak trick
\divide\figboxht by\baselineskip%      how many lines?
\global\figboxlines=\figboxht
%
\global\figboxwid=\wd\figurebox
\global\advance\figboxwid by \figboxmargin
\global\advance\figboxwid by -\figboxhang
\setmypar\noindent}
%

%
% Useful macros
%
% for lengthening (and shortening) picture space
%
\def\addlines#1{\global\advance\figboxlines by #1\myparshape}
\def\zerolines{\origpar\global\figboxlines=0\myparshape}

%
% for passing any marginal figures
%
\def\passpict{\par\ifnum\figboxlines>1\vskip\figboxlines\baselineskip
\zerolines\fi}

\def\emptybox#1#2{\hbox to #1{\vbox to #2{\vss}\hss}}

%%%%%%%%%%%%%%%%%%%%%
%\par hacking  
% We define our own version of Tex's \par and tell Latex it is the original

\global\let\origpar=\@@par%     --- Latex's saved version of Tex's \par
\global\let\dopar=\origpar
\global\def\@@par{\dopar}
\@setpar{\dopar}

\def\setmypar{\global\let\dopar=\mypar
\global\prevgraf=0\myparshape}

\def\mypar{\origpar\global\advance\figboxlines by -\prevgraf%
\global\prevgraf=0\myparshape}

\def\myparshape{\relax%
\ifnum\figboxlines>1\theparshape \else
\global\hangindent=0pt\global\hangafter=1%  these sometimes get restored...
\global\let\dopar=\origpar\fi}

\def\theparshape{%
\ifisleftbox\global\hangindent=-\figboxwid 
\else\global\hangindent=\figboxwid \fi
\global\hangafter=-\figboxlines \global\advance\hangafter by 1%
}


%%%%%%%%%%%%%%%%
%
%  Macros to make the list of figures and captions used by journals
%
%  To use, execute \journalpicts{scalefactor} at the beginning of the
%  file.  All figbox figures and tables will be written to a .pic file
%  and all captions to a .cap file.  On \enddocument, these files are
%  read in and the lists of figures and captions are produced.
%  
%  For non-english versions, or to add other types of floats, the
%  macro \definefnum should be changed.
%

\def\definefnum#1{%             --- ugly latex hacking to get the
%                                       right figure/table name 
\def\fnum@figure{Figure \ref{#1}}%
\def\fnum@table{Table \ref{#1}}%
\def\fnum@code{Algorithm \ref{#1}}%
}

\def\writepict#1{}
\def\writecapt#1{}

%%%%%%%%%%%%%%%%
% \journalpicts{globalscalefactor} writes all pictures and captions to
%       files and modifies \enddocument to read these files and
%       produce lists of enlarged figures and of captions
% 

\def\journalpicts#1{%           global scale factor
\newwrite\pictfile
\newwrite\captfile
%
\openout\pictfile\jobname.pic
\openout\captfile\jobname.cap
%
\gdef\writepict##1{\unexpandedwrite\pictfile{\doit##1}}%
\gdef\writecapt##1{\unexpandedwrite\captfile{\doit##1}}%
%
\global\let\ENDdocument=\enddocument
\gdef\enddocument{\DOjournalpicts{#1}\ENDdocument}
}

%
%       generate enlarged figures 1 per page and list captions
%
\def\DOjournalpicts#1{{%
\def\writepict##1{}\closeout\pictfile
\def\writecapt##1{}\closeout\captfile
\@fileswfalse%                  --- turn off toc, lof, and so on.
%
\onecolumn
\def\globalscaling{#1}
\def\doit##1##2##3##4{%         --- print a picture on blank page
%                                       {box}{style}{label}{caption}
\figboxaux[t]{\hss##1\hss}{##2}{}{}%
\vspace*{1in}
\definefnum{##3}%               --- use Figure \ref{label} or Table \ref{label}
\captionandlabel{##2}{}{##4}
\clearpage}%
%
\input\jobname.pic
%
\def\doit##1##2##3{%            --- print caption
%                                       {style}{label}{caption}
\definefnum{##2}%               --- use Figure \ref{label} or Table \ref{label}
\captionandlabel{##1}{}{##3}}%
%
\raggedright\let\captionstyle=\raggedright
\def\@makecaption##1##2{##1: ##2\par}%  --- Be sure captions aren't centered
%
\input\jobname.cap
}}%                             --- to be followed by \enddocument



%%%%%%%%%%%%%%%%
%
%  Box macros
%
%%%%%%%%%%%%%%%%
%
% \boxeps[local commands] {file} {scale}
%
% options \DVIscaling \figuredirectory \scale \horizscale \vertscale
% \missingfigure \globalscaling
%
% \boxps[local commands] {file} {scale} {width} {height}
%
% \scale \horizscale \vertscale 

%
%  Various `boxers' --- which one is used depends on where the origin
%  of the picture is.  If you are using the bounding boxes of
%  encapsulated postscript, just use the default \llboxer.  Otherwise,
%  you have to figure out where the origin is and set \boxer appropriately.
%

\def\llboxer#1{\vbox to \figboxht{\vfil\hbox to \figboxwid{#1\hfill}}}
\def\lcboxer#1{\vbox to \figboxht{\vfil\hbox to \figboxwid{\hfill#1\hfill}}}
\def\oldboxer#1{\vbox to \figboxht{\vfil
                      \hbox to \figboxwid{\hfill\llap{#1\hskip4.25in}\hfill}}}
\def\ulboxer#1{\vbox to \figboxht{\hbox to \figboxwid{#1\hfill}\vfil}}
\def\ccboxer#1{\vbox to \figboxht{\vfil
                        \hbox to \figboxwid{\hfill#1\hfill}\vfil}}

% a macro to strip the pt off a dimension
%
{\catcode`\p=12\catcode`\t=12
\gdef\removedimen#1pt{#1}}

\def\defscaled#1#2{#2=\DVIscaling#2%
\xdef#1{\expandafter\removedimen\the#2}}

\def\DVIspace{ }
\newdimen\hscalefactor
\newdimen\vscalefactor

\def\scale#1{\horizscale{#1}\vertscale{#1}}
\def\horizscale#1{\hscalefactor=#1\hscalefactor\figboxht=#1\figboxht}
\def\vertscale#1{\vscalefactor=#1\vscalefactor\figboxwid=#1\figboxwid}

%%%%%%%%%%%%%%%%
%

\def\boxps{%
\@ifnextchar[{\boxpsaux}{\boxpsaux[\relax]}}

\def\boxpsaux[#1]#2#3#4#5{%
{\figboxwid#4\figboxht#5\hscalefactor=1pt\vscalefactor=1pt%
\scale{#3}%
\scale{\globalscaling}%
#1%
\defscaled\DVIhscale\hscalefactor
\defscaled\DVIvscale\vscalefactor
\boxer{\special{psfile=\figpsfilename\DVIspace
                hscale=\DVIhscale\DVIspace 
                vscale=\DVIvscale}}}%
}


%%%%%%%%%%%%%%%%
% The following is from tex/inputs/epsf.tex by Tom Rokicki
% with slight modifications by Jack Snoeyink.
%
%   Written by Tomas Rokicki of Radical Eye Software, 29 Mar 1989.
%
%   Include an ecapsulated postscript graphic.
%   Works by finding the bounding box comment.
%
\newread\Epsffilein
\newif\ifEpsffileok
\newif\ifEpsfbbfound

\newdimen\pspoints
\pspoints=1in
\divide\pspoints by 72


\def\boxeps{%
\@ifnextchar[{\boxepsaux}{\boxepsaux[\relax]}}
\def\boxepsaux[#1]#2#3{%
%
%   The first thing we need to do is to open the
%   PostScript file, if possible.
%
%\tracingall
\gdef\figpsfilename{\figuredirectory/#2.ps}
\openin\Epsffilein=\figuredirectory/#2.ps
\ifeof\Epsffilein 
\gdef\figpsfilename{\figuredirectory/#2.eps}
\openin\Epsffilein=\figuredirectory/#2.eps \fi
\ifeof\Epsffilein\message{I couldn't open \figuredirectory/#2.ps or \figpsfilename}%
\missingfigure{#2}
\else
%
%   Okay, we got it.  Now we scan lines until we find one
%   that doesn't start with a percentage sign.  We're
%   looking for the bounding box comment.
%
   {\Epsffileoktrue\Epsfbbfoundfalse
    \catcode`\%=11 \catcode`\\=11
    \catcode`\{=11 \catcode`\}=11
    \catcode`\$=11 \catcode`\^=11
    \catcode`\&=11 \catcode`\#=11
    \catcode`\~=11 \catcode`\_=11
    \loop
       \read\Epsffilein to \Epsffileline
       \ifeof\Epsffilein\Epsffileokfalse\else
%
%   Now we check to make sure the first character is a % sign,
%   and that the rest are `%BoundingBox:' if necessary.
%
          \expandafter\Epsfaux\Epsffileline . .\\%
       \fi
   \ifEpsffileok\repeat
   \ifEpsfbbfound
        \figboxht=\Epsfury\pspoints
        \advance\figboxht by-\Epsflly\pspoints
        \figboxwid=\Epsfurx\pspoints
        \advance\figboxwid by-\Epsfllx\pspoints
   \else
        \message{No bounding box comment in \figpsfilename }%
        \figboxwid=2in\figboxht=1in%
   \fi%
   \immediate\closein\Epsffilein
   \hscalefactor=1pt\vscalefactor=1pt%
   \scale{#3}%
   \scale{\globalscaling}%
   #1%
%
   \defscaled\DVIhscale\hscalefactor
   \defscaled\DVIvscale\vscalefactor
%
   \hscalefactor=-\Epsfllx\hscalefactor
   \hscalefactor=1.00375\hscalefactor%  --- 72.27/72 = TeXpoints/PSpoints
   \defscaled\DVIhoffset\hscalefactor
%
   \vscalefactor=-\Epsflly\vscalefactor
   \vscalefactor=1.00375\vscalefactor%  --- 72.27/72 = TeXpoints/PSpoints
   \defscaled\DVIvoffset\vscalefactor
%
   \llboxer{\special{psfile=\figpsfilename\DVIspace
                hoffset=\DVIhoffset\DVIspace
                voffset=\DVIvoffset\DVIspace
                hscale=\DVIhscale\DVIspace 
                vscale=\DVIvscale}} }%
\fi
}%
%
%   We need these for comparison purposes.
%
{\catcode`\%=11 \global\let\Epsfpar=\par
\global\let\Epsfpercent=%\global\def\Epsfbblit{%BoundingBox:}}%
%
%   This is our function that checks for `%BoundingBox:' and grabs the
%   values if they are found.
%
\long\def\Epsfaux#1#2 #3\\{\relax\ifx#1\Epsfpercent
   \def\testit{#2}\ifx\testit\Epsfbblit
      \Epsfsize #3 . . . .\\%
      \global\Epsffileokfalse
      \global\Epsfbbfoundtrue
   \fi\else\ifx#1\Epsfpar\else\global\Epsffileokfalse\fi\fi}%
%
%   Here we grab the values and stuff them in the appropriate definitions.
%
\def\Epsfsize#1 #2 #3 #4 #5\\{\global\def\Epsfllx{#1}\global\def\Epsflly{#2}%
   \global\def\Epsfurx{#3}\global\def\Epsfury{#4}}%




\catcode`\@=12                  % done with with internals

%%%%%%%%%%%%%%%%%%%%%%%%%%%%%%%%%%%%%%%%%%%%%%%%%%%%%%%%%%%%%%%%
%
% For backward compatability
%
% \pic file; height in inches; width in inches; caption\par
% \picsc file; ht (in); wid (in); scale; caption\par
%
% \mpic file; height in inches; width in inches; caption\par
% \mpicsc file; ht (in); wid (in); scale; caption\par
%
%%%%%%%%%%%%%%%%

\def\pic#1;#2;#3;#4\par{\picsc#1;#2;#3;1;#4\par}% file; wd; ht; caption 

\def\picsc#1;#2;#3;#4;#5\par{% file; wd; ht; scale; caption
\figbox[htb]{\boxeps{#1}{#4}%{#2in}{#3in}%
}{figure}{#1}{%
% (#1: #2x#3 sc #4) %                   Uncomment for debugging
#5}}


\def\mpic#1;#2;#3;#4\par{\mpicsc#1;#2;#3;1;#4\par}% file; wd; ht; caption 

\def\mpicsc#1;#2;#3;#4;#5\par{% file; wd; ht; scale; caption
\figbox[l]{\boxeps{#1}{#4}%{#2in}{#3in}%
}{figure}{#1}{%
% (#1: #2x#3 sc #4) %                   Uncomment for debugging
#5}}

